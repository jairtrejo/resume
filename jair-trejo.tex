%%%%%%%%%%%%%%%%%%%%%%%%%%%%%%%%%%%%%%%%%
% Friggeri Resume/CV
% XeLaTeX Template
% Version 1.0 (5/5/13)
%
% This template has been downloaded from:
% http://www.LaTeXTemplates.com
%
% Original author:
% Adrien Friggeri (adrien@friggeri.net)
% https://github.com/afriggeri/CV
%
% License:
% CC BY-NC-SA 3.0 (http://creativecommons.org/licenses/by-nc-sa/3.0/)
%
% Important notes:
% This template needs to be compiled with XeLaTeX and the bibliography, if used,
% needs to be compiled with biber rather than bibtex.
%
%%%%%%%%%%%%%%%%%%%%%%%%%%%%%%%%%%%%%%%%%

\documentclass[]{friggeri-cv} % Add 'print' as an option into the square bracket to remove colors from this template for printing

\usepackage{fontawesome}
\newfontfamily{\FA}{FontAwesome Regular}
\def\home{{\FA \faHome}}
\def\github{{\FA \faGithub}}
\def\coderwall{{\FA \faCog}}
\def\linkedin{{\FA \faLinkedinSign}}
\def\mystar{{\FA \faStar}}

\addbibresource{bibliography.bib} % Specify the bibliography file to include publications

\begin{document}

\header{Jair}{Trejo}{Web development and consulting} % Your name and current job title/field

%----------------------------------------------------------------------------------------
%	SIDEBAR SECTION
%----------------------------------------------------------------------------------------

\begin{aside} % In the aside, each new line forces a line break
\section{contact}
+52 155 3931 3400
~
\href{mailto:jair@jairtrejo.mx}{jair@jairtrejo.mx}
\href{http://jairtrejo.mx}{http://jairtrejo.mx}
\section{languages}
native spanish
fluent in english
intermediate japanese
\section{programming}
{\color{red} \mystar} Python
Javascript, Clojure
CSS3 \& HTML5
\section{technologies}
Django
AngularJS
Git, Jenkins, Vagrant
\end{aside}

Web project developer, manager and consultant. I have been developing software professionally for the past four years, mostly on the web, but my interest in programming is a life-long affair.

%----------------------------------------------------------------------------------------
%	WORK EXPERIENCE SECTION
%----------------------------------------------------------------------------------------

\section{experience}

\begin{entrylist}
\input{experience}
%------------------------------------------------
\end{entrylist}

%----------------------------------------------------------------------------------------
%	EDUCATION SECTION
%----------------------------------------------------------------------------------------

\section{education}

\begin{entrylist}
\input{education}
%------------------------------------------------
\end{entrylist}

%----------------------------------------------------------------------------------------
%	TALKS AND PRESENTATIONS SECTION
%----------------------------------------------------------------------------------------

\section{talks and presentations}

\begin{entrylist}
%------------------------------------------------
\entry
{2013}
{Fabric for fun and profit}
{PyCon Asia Pacific 2013, Tokyo}
{Presentation on Fabric, a Python tool for performing systems administration tasks through ssh. It introduces the tool and explains how we use it for Vagrant development at Vinco Orbis.}
%------------------------------------------------
\end{entrylist}

%----------------------------------------------------------------------------------------
%	RELEVANT LINKS SECTION
%----------------------------------------------------------------------------------------

\section{relevant links}

\begin{entrylist}
%------------------------------------------------
\entry
{Home page}
{\home\enspace Home page and blog}
{}
{http://jairtrejo.mx}
%------------------------------------------------
\entry
{Github}
{\github\enspace Open source projects}
{}
{http://github.com/jairtrejo}
%------------------------------------------------
\entry
{Coderwall}
{\coderwall\enspace Short web development tips}
{}
{http://coderwall.com/jairtrejo}
%------------------------------------------------
\entry
{Linked In}
{\linkedin\enspace Professional information and networking}
{}
{http://mx.linkedin.com/in/jairtrejo}
%------------------------------------------------
\end{entrylist}

%----------------------------------------------------------------------------------------
%	PROJECTS SECTION
%----------------------------------------------------------------------------------------
\section{featured projects}

\begin{entrylist}
%------------------------------------------------
\entry
{2013}
{Prestigos}
{http://prestigos.com}
{Development of a complex client-side application for the recording and analysis of employee performance over time. Developed using AngularJS and a Django backend.}
%------------------------------------------------
\entry
{2013}
{Database of Competition Law rulings}
{http://crc.jairtrejo.mx}
{Work in progress. Consulting, coordination and user experience design for a database of rulings on competition law matters accross Latin America. Built for the \emph{Centro Regional de Competencia para América Latina}.

I explored the provided taxonomy for the rulings and defined a CouchDB database structure for storing them. I also coordinated the development effort by a team of three people, and designed the user interface for the user-facing side of the project, including a directory of all rulings and a search interface, according to the needs of our clients.}
%------------------------------------------------
%------------------------------------------------
\entry
{2012}
{Datoz}
{http://datoz.com}
{Development of a real estate information browser. It provides a backend for entering detailed, bilingual information on offices, industrial properties and tenants along with their geographical position; and a frontend for real estate agents to search for them by geographical area.

Some interesting aspects of the project are: Integration with Google Maps for input (polygon search) and output (showing buildings on the map), internationalization (both the interface and content are bilingual) and using PostGIS for geographical information storage and retrieval}
%------------------------------------------------
\entry
{2012}
{TuOla}
{http://tuola.mx}
{Development of a rewards system for “green” activities. Users can register some “green programs” on the site (public bicycle lending program, garbage recycling, etc.) and they receive points every time they use them, which they can exchange for rewards.

As the sole developer for the first part of the project I added social network integration, some image processing and learned to use Twitter Bootstrap.

After the initial phase, I coordinated a development team to grow the site and add more features.}
%------------------------------------------------
\end{entrylist}

%----------------------------------------------------------------------------------------
%	INTERESTS SECTION
%----------------------------------------------------------------------------------------

\section{interests}

\textbf{professional:} Web development, user experience, client-side applications, distributed systems, functional programming, machine learning, data visualization \\
\textbf{personal:} literature, japanese language, travel, mathematics

%----------------------------------------------------------------------------------------

\end{document}
